% Options for packages loaded elsewhere
\PassOptionsToPackage{unicode}{hyperref}
\PassOptionsToPackage{hyphens}{url}
%
\documentclass[
]{article}
\usepackage{amsmath,amssymb}
\usepackage{iftex}
\ifPDFTeX
  \usepackage[T1]{fontenc}
  \usepackage[utf8]{inputenc}
  \usepackage{textcomp} % provide euro and other symbols
\else % if luatex or xetex
  \usepackage{unicode-math} % this also loads fontspec
  \defaultfontfeatures{Scale=MatchLowercase}
  \defaultfontfeatures[\rmfamily]{Ligatures=TeX,Scale=1}
\fi
\usepackage{lmodern}
\ifPDFTeX\else
  % xetex/luatex font selection
\fi
% Use upquote if available, for straight quotes in verbatim environments
\IfFileExists{upquote.sty}{\usepackage{upquote}}{}
\IfFileExists{microtype.sty}{% use microtype if available
  \usepackage[]{microtype}
  \UseMicrotypeSet[protrusion]{basicmath} % disable protrusion for tt fonts
}{}
\makeatletter
\@ifundefined{KOMAClassName}{% if non-KOMA class
  \IfFileExists{parskip.sty}{%
    \usepackage{parskip}
  }{% else
    \setlength{\parindent}{0pt}
    \setlength{\parskip}{6pt plus 2pt minus 1pt}}
}{% if KOMA class
  \KOMAoptions{parskip=half}}
\makeatother
\usepackage{xcolor}
\usepackage[margin=1in]{geometry}
\usepackage{color}
\usepackage{fancyvrb}
\newcommand{\VerbBar}{|}
\newcommand{\VERB}{\Verb[commandchars=\\\{\}]}
\DefineVerbatimEnvironment{Highlighting}{Verbatim}{commandchars=\\\{\}}
% Add ',fontsize=\small' for more characters per line
\usepackage{framed}
\definecolor{shadecolor}{RGB}{248,248,248}
\newenvironment{Shaded}{\begin{snugshade}}{\end{snugshade}}
\newcommand{\AlertTok}[1]{\textcolor[rgb]{0.94,0.16,0.16}{#1}}
\newcommand{\AnnotationTok}[1]{\textcolor[rgb]{0.56,0.35,0.01}{\textbf{\textit{#1}}}}
\newcommand{\AttributeTok}[1]{\textcolor[rgb]{0.13,0.29,0.53}{#1}}
\newcommand{\BaseNTok}[1]{\textcolor[rgb]{0.00,0.00,0.81}{#1}}
\newcommand{\BuiltInTok}[1]{#1}
\newcommand{\CharTok}[1]{\textcolor[rgb]{0.31,0.60,0.02}{#1}}
\newcommand{\CommentTok}[1]{\textcolor[rgb]{0.56,0.35,0.01}{\textit{#1}}}
\newcommand{\CommentVarTok}[1]{\textcolor[rgb]{0.56,0.35,0.01}{\textbf{\textit{#1}}}}
\newcommand{\ConstantTok}[1]{\textcolor[rgb]{0.56,0.35,0.01}{#1}}
\newcommand{\ControlFlowTok}[1]{\textcolor[rgb]{0.13,0.29,0.53}{\textbf{#1}}}
\newcommand{\DataTypeTok}[1]{\textcolor[rgb]{0.13,0.29,0.53}{#1}}
\newcommand{\DecValTok}[1]{\textcolor[rgb]{0.00,0.00,0.81}{#1}}
\newcommand{\DocumentationTok}[1]{\textcolor[rgb]{0.56,0.35,0.01}{\textbf{\textit{#1}}}}
\newcommand{\ErrorTok}[1]{\textcolor[rgb]{0.64,0.00,0.00}{\textbf{#1}}}
\newcommand{\ExtensionTok}[1]{#1}
\newcommand{\FloatTok}[1]{\textcolor[rgb]{0.00,0.00,0.81}{#1}}
\newcommand{\FunctionTok}[1]{\textcolor[rgb]{0.13,0.29,0.53}{\textbf{#1}}}
\newcommand{\ImportTok}[1]{#1}
\newcommand{\InformationTok}[1]{\textcolor[rgb]{0.56,0.35,0.01}{\textbf{\textit{#1}}}}
\newcommand{\KeywordTok}[1]{\textcolor[rgb]{0.13,0.29,0.53}{\textbf{#1}}}
\newcommand{\NormalTok}[1]{#1}
\newcommand{\OperatorTok}[1]{\textcolor[rgb]{0.81,0.36,0.00}{\textbf{#1}}}
\newcommand{\OtherTok}[1]{\textcolor[rgb]{0.56,0.35,0.01}{#1}}
\newcommand{\PreprocessorTok}[1]{\textcolor[rgb]{0.56,0.35,0.01}{\textit{#1}}}
\newcommand{\RegionMarkerTok}[1]{#1}
\newcommand{\SpecialCharTok}[1]{\textcolor[rgb]{0.81,0.36,0.00}{\textbf{#1}}}
\newcommand{\SpecialStringTok}[1]{\textcolor[rgb]{0.31,0.60,0.02}{#1}}
\newcommand{\StringTok}[1]{\textcolor[rgb]{0.31,0.60,0.02}{#1}}
\newcommand{\VariableTok}[1]{\textcolor[rgb]{0.00,0.00,0.00}{#1}}
\newcommand{\VerbatimStringTok}[1]{\textcolor[rgb]{0.31,0.60,0.02}{#1}}
\newcommand{\WarningTok}[1]{\textcolor[rgb]{0.56,0.35,0.01}{\textbf{\textit{#1}}}}
\usepackage{graphicx}
\makeatletter
\newsavebox\pandoc@box
\newcommand*\pandocbounded[1]{% scales image to fit in text height/width
  \sbox\pandoc@box{#1}%
  \Gscale@div\@tempa{\textheight}{\dimexpr\ht\pandoc@box+\dp\pandoc@box\relax}%
  \Gscale@div\@tempb{\linewidth}{\wd\pandoc@box}%
  \ifdim\@tempb\p@<\@tempa\p@\let\@tempa\@tempb\fi% select the smaller of both
  \ifdim\@tempa\p@<\p@\scalebox{\@tempa}{\usebox\pandoc@box}%
  \else\usebox{\pandoc@box}%
  \fi%
}
% Set default figure placement to htbp
\def\fps@figure{htbp}
\makeatother
\setlength{\emergencystretch}{3em} % prevent overfull lines
\providecommand{\tightlist}{%
  \setlength{\itemsep}{0pt}\setlength{\parskip}{0pt}}
\setcounter{secnumdepth}{-\maxdimen} % remove section numbering
\usepackage{bookmark}
\IfFileExists{xurl.sty}{\usepackage{xurl}}{} % add URL line breaks if available
\urlstyle{same}
\hypersetup{
  hidelinks,
  pdfcreator={LaTeX via pandoc}}

\author{}
\date{\vspace{-2.5em}}

\begin{document}

\section{Lesson 3: Introduction to
Packages}\label{lesson-3-introduction-to-packages}

\begin{itemize}
\tightlist
\item
  What are R packages and why are they important?
\item
  Installing packages from CRAN (\texttt{install.packages()}).
\item
  Loading packages into your R session (\texttt{library()},
  \texttt{require()}).
\item
  Exploring package documentation.
\end{itemize}

While R's base installation is powerful, its true strength lies in its
extensibility through \textbf{packages}. Packages are collections of
functions, data, and compiled code in a well-defined format, designed to
extend R's capabilities for specific tasks. They are the primary way the
R community shares reusable code and functionalities.

\subsection{\texorpdfstring{\textbf{Phase 1: Introduction to R and
Fundamentals}}{Phase 1: Introduction to R and Fundamentals}}\label{phase-1-introduction-to-r-and-fundamentals}

\subsubsection{\texorpdfstring{\textbf{Module 1.4: Functions and
Packages}}{Module 1.4: Functions and Packages}}\label{module-1.4-functions-and-packages}

\begin{center}\rule{0.5\linewidth}{0.5pt}\end{center}

\subsubsection{\texorpdfstring{\textbf{Lesson 3: Introduction to
Packages}}{Lesson 3: Introduction to Packages}}\label{lesson-3-introduction-to-packages-1}

This lesson will introduce you to the concept of R packages, why they
are indispensable for data analysis, and how to effectively manage them.
You'll learn how to install, load, and find packages, as well as get an
overview of some of the most commonly used and powerful packages in the
R ecosystem.

\begin{center}\rule{0.5\linewidth}{0.5pt}\end{center}

\paragraph{\texorpdfstring{\textbf{1. What are R Packages and Why are
They
Used?}}{1. What are R Packages and Why are They Used?}}\label{what-are-r-packages-and-why-are-they-used}

\begin{itemize}
\tightlist
\item
  \textbf{Definition:} An R package is a collection of
  \texttt{functions}, data, and documentation that extends the
  capabilities of R's base system. Think of them as add-ons or plugins
  for R.
\item
  \textbf{Purpose:}

  \begin{itemize}
  \tightlist
  \item
    \textbf{Extend Functionality:} Provide specialized tools for
    statistics, machine learning, visualization, data manipulation, web
    scraping, and more.
  \item
    \textbf{Code Reusability:} Prevent users from reinventing the wheel
    by providing pre-written, tested, and optimized code for common
    tasks.
  \item
    \textbf{Community Contribution:} Allow R users and developers
    worldwide to share their work and contribute to the R ecosystem.
  \item
    \textbf{Organization:} Group related functions and data together,
    making them easier to discover and use.
  \end{itemize}
\end{itemize}

R's core functionality is powerful, but packages are what truly make R a
versatile tool for almost any data-related task. The Comprehensive R
Archive Network (CRAN) hosts over 20,000 packages, with thousands more
available elsewhere (e.g., Bioconductor, GitHub).

\begin{center}\rule{0.5\linewidth}{0.5pt}\end{center}

\paragraph{\texorpdfstring{\textbf{2. Installing Packages
(\texttt{install.packages()})}}{2. Installing Packages (install.packages())}}\label{installing-packages-install.packages}

Before you can use a package, you need to install it onto your system.
This typically only needs to be done once per package on a given R
installation.

\textbf{Syntax:}

install.packages(``package\_name'')

\textbf{Parameters:}

\begin{itemize}
\tightlist
\item
  \texttt{"package\_name"}: A character string specifying the name of
  the package you want to install. It \textbf{must} be enclosed in
  quotes.
\end{itemize}

\textbf{Important Notes:}

\begin{itemize}
\tightlist
\item
  You need an active internet connection to install packages from CRAN.
\item
  You might be prompted to choose a CRAN mirror (a server from which to
  download). Choose one geographically close to you for faster
  downloads.
\item
  \texttt{install.packages()} downloads the package from CRAN (or
  another specified repository) and places it in your R library folder.
\end{itemize}

\textbf{Code Snippets:}

\begin{Shaded}
\begin{Highlighting}[]
\FunctionTok{print}\NormalTok{(}\StringTok{"{-}{-}{-} Installing Packages {-}{-}{-}"}\NormalTok{)}
\end{Highlighting}
\end{Shaded}

\begin{verbatim}
## [1] "--- Installing Packages ---"
\end{verbatim}

\begin{Shaded}
\begin{Highlighting}[]
\CommentTok{\# Example: Installing a popular data manipulation package, \textquotesingle{}dplyr\textquotesingle{}}
\CommentTok{\# If you\textquotesingle{}ve installed it before, R will usually just say it\textquotesingle{}s already installed.}
\CommentTok{\# It\textquotesingle{}s good practice to check if it\textquotesingle{}s installed first, though for a learning}
\CommentTok{\# environment, simply running install.packages() is fine.}

\CommentTok{\# You can check if a package is installed:}
\CommentTok{\# if (!requireNamespace("dplyr", quietly = TRUE)) \{}
\CommentTok{\#   install.packages("dplyr")}
\CommentTok{\# \}}

\CommentTok{\# Let\textquotesingle{}s install a less common example for demonstration without affecting}
\CommentTok{\# your existing setup too much, or use a package that is commonly used.}
\CommentTok{\# \textquotesingle{}stringr\textquotesingle{} is a good basic text manipulation package.}

\CommentTok{\# Installing \textquotesingle{}stringr\textquotesingle{}}
\CommentTok{\# install.packages("stringr") \# Uncomment to run}

\CommentTok{\# To see where packages are installed (your R library paths):}
\FunctionTok{print}\NormalTok{(}\StringTok{"Your R library paths:"}\NormalTok{)}
\end{Highlighting}
\end{Shaded}

\begin{verbatim}
## [1] "Your R library paths:"
\end{verbatim}

\begin{Shaded}
\begin{Highlighting}[]
\FunctionTok{print}\NormalTok{(}\FunctionTok{.libPaths}\NormalTok{())}
\end{Highlighting}
\end{Shaded}

\begin{verbatim}
## [1] "C:/Users/Titus/AppData/Local/R/win-library/4.4"
## [2] "C:/Program Files/R/R-4.4.3/library"
\end{verbatim}

\begin{Shaded}
\begin{Highlighting}[]
\FunctionTok{print}\NormalTok{(}\StringTok{"Installation successful (or package already installed) if no errors occurred."}\NormalTok{)}
\end{Highlighting}
\end{Shaded}

\begin{verbatim}
## [1] "Installation successful (or package already installed) if no errors occurred."
\end{verbatim}

\paragraph{\texorpdfstring{\textbf{3. Loading Packages
(\texttt{library()},
\texttt{require()})}}{3. Loading Packages (library(), require())}}\label{loading-packages-library-require}

Installing a package makes it available on your system, but to use its
functions in your current R session, you must \textbf{load} it. This
typically needs to be done at the beginning of each R session where you
intend to use functions from that package.

\begin{itemize}
\tightlist
\item
  \texttt{library(package\_name)}: The most common and recommended way
  to load a package. It loads the package into your current R session.
  If the package isn't found, it will produce an error and stop
  execution.
\item
  \texttt{require(package\_name)}: Similar to \texttt{library()}, but it
  returns \texttt{TRUE} if the package is successfully loaded and
  \texttt{FALSE} if not (without stopping execution). It's often used
  within functions or scripts where you want to conditionally load a
  package or check for its availability.
\end{itemize}

\textbf{Syntax:}

library(package\_name) require(package\_name) \# Less common for
interactive use

\textbf{Parameters:}

\begin{itemize}
\tightlist
\item
  \texttt{package\_name}: The name of the package to load. \textbf{No
  quotes are generally needed} for \texttt{library()} or
  \texttt{require()}, although they work with quotes too.
\end{itemize}

\textbf{Code Snippets:}

\begin{Shaded}
\begin{Highlighting}[]
\FunctionTok{print}\NormalTok{(}\StringTok{"{-}{-}{-} Loading Packages {-}{-}{-}"}\NormalTok{)}
\end{Highlighting}
\end{Shaded}

\begin{verbatim}
## [1] "--- Loading Packages ---"
\end{verbatim}

\begin{Shaded}
\begin{Highlighting}[]
\CommentTok{\# Let\textquotesingle{}s use the \textquotesingle{}stringr\textquotesingle{} package which you might have installed in the previous step}
\CommentTok{\# or is likely already installed on most R installations.}

\CommentTok{\# Load the stringr package}
\FunctionTok{library}\NormalTok{(stringr)}
\FunctionTok{print}\NormalTok{(}\StringTok{"stringr package loaded using library()."}\NormalTok{)}
\end{Highlighting}
\end{Shaded}

\begin{verbatim}
## [1] "stringr package loaded using library()."
\end{verbatim}

\begin{Shaded}
\begin{Highlighting}[]
\CommentTok{\# Now you can use functions from stringr, e.g., str\_length()}
\NormalTok{text\_data }\OtherTok{\textless{}{-}} \StringTok{"Introduction to R Packages"}
\NormalTok{char\_count }\OtherTok{\textless{}{-}} \FunctionTok{str\_length}\NormalTok{(text\_data)}
\FunctionTok{print}\NormalTok{(}\FunctionTok{paste}\NormalTok{(}\StringTok{"Number of characters in \textquotesingle{}"}\NormalTok{, text\_data, }\StringTok{"\textquotesingle{}:"}\NormalTok{, char\_count))}
\end{Highlighting}
\end{Shaded}

\begin{verbatim}
## [1] "Number of characters in ' Introduction to R Packages ': 26"
\end{verbatim}

\begin{Shaded}
\begin{Highlighting}[]
\CommentTok{\# Using require()}
\CommentTok{\# Usually used in conditional checks, e.g.,}
\CommentTok{\# if (require(ggplot2)) \{}
\CommentTok{\#   print("ggplot2 is available and loaded.")}
\CommentTok{\# \} else \{}
\CommentTok{\#   print("ggplot2 is not available. Please install it.")}
\CommentTok{\# \}}

\CommentTok{\# Example of a package that might not be installed, to demonstrate require()}
\CommentTok{\# This will likely return FALSE and print the \textquotesingle{}else\textquotesingle{} message}
\ControlFlowTok{if}\NormalTok{ (}\FunctionTok{require}\NormalTok{(nonexistentpackage123, }\AttributeTok{quietly =} \ConstantTok{TRUE}\NormalTok{)) \{}
  \FunctionTok{print}\NormalTok{(}\StringTok{"nonexistentpackage123 is available and loaded."}\NormalTok{)}
\NormalTok{\} }\ControlFlowTok{else}\NormalTok{ \{}
  \FunctionTok{print}\NormalTok{(}\StringTok{"nonexistentpackage123 is not available or could not be loaded."}\NormalTok{)}
\NormalTok{\}}
\end{Highlighting}
\end{Shaded}

\begin{verbatim}
## [1] "nonexistentpackage123 is not available or could not be loaded."
\end{verbatim}

\textbf{Note:} Once a package is loaded, its functions are available
directly by name. If two loaded packages have functions with the same
name, R will use the one from the package loaded \emph{last}. You can
always explicitly specify a function's package using
\texttt{package\_name::function\_name()}. For example,
\texttt{dplyr::filter()} if you want to ensure you're using
\texttt{dplyr}'s filter function.

\begin{center}\rule{0.5\linewidth}{0.5pt}\end{center}

\paragraph{\texorpdfstring{\textbf{4. Finding and Exploring
Packages}}{4. Finding and Exploring Packages}}\label{finding-and-exploring-packages}

The R community has created an enormous number of packages. Knowing
where to find them and how to explore their functionalities is crucial.

\begin{itemize}
\tightlist
\item
  \textbf{CRAN (Comprehensive R Archive Network):} The official
  repository for R packages.

  \begin{itemize}
  \tightlist
  \item
    \textbf{Website:} \url{https://cran.r-project.org/web/packages/}
  \item
    You can browse packages by name, task view (e.g., ``Machine
    Learning'', ``Time Series''), or author. Each package has a
    dedicated page with a description, downloads, and links to
    documentation.
  \end{itemize}
\item
  \textbf{Bioconductor:} A specialized repository for bioinformatics and
  computational biology packages.

  \begin{itemize}
  \tightlist
  \item
    \textbf{Website:} \url{https://www.bioconductor.org/}
  \end{itemize}
\item
  \textbf{GitHub:} Many developers host packages on GitHub before or
  instead of submitting to CRAN. You can install directly from GitHub
  using the \texttt{install\_github()} function from the
  \texttt{devtools} package.
\item
  \textbf{R Documentation:} Once a package is loaded, you can access its
  documentation:

  \begin{itemize}
  \tightlist
  \item
    \texttt{?function\_name}: Opens the help page for a specific
    function.
  \item
    \texttt{help(package\ =\ "package\_name")}: Opens the main help page
    for the entire package.
  \item
    \texttt{example(function\_name)}: Runs the examples provided in the
    function's help page.
  \end{itemize}
\end{itemize}

\textbf{Code Snippets:}

\begin{Shaded}
\begin{Highlighting}[]
\FunctionTok{print}\NormalTok{(}\StringTok{"{-}{-}{-} Finding and Exploring Packages {-}{-}{-}"}\NormalTok{)}
\end{Highlighting}
\end{Shaded}

\begin{verbatim}
## [1] "--- Finding and Exploring Packages ---"
\end{verbatim}

\begin{Shaded}
\begin{Highlighting}[]
\CommentTok{\# Access help for a function (e.g., from base R)}
\CommentTok{\# ?mean \# Uncomment to open help page for \textquotesingle{}mean\textquotesingle{}}

\CommentTok{\# Access help for a function from a loaded package (e.g., stringr::str\_length)}
\CommentTok{\# ?str\_length \# Uncomment to open help page for \textquotesingle{}str\_length\textquotesingle{}}

\CommentTok{\# View all functions/objects in a loaded package}
\CommentTok{\# ls("package:stringr") \# Uncomment to list all objects in stringr namespace}

\CommentTok{\# Run examples from a function\textquotesingle{}s help page}
\CommentTok{\# example(str\_length) \# Uncomment to run examples for str\_length}
\end{Highlighting}
\end{Shaded}

\textbf{Expected (if uncommented):} (Output would appear in R's help
viewer or console)

\begin{center}\rule{0.5\linewidth}{0.5pt}\end{center}

\paragraph{\texorpdfstring{\textbf{5. Commonly Used
Packages}}{5. Commonly Used Packages}}\label{commonly-used-packages}

Here are some essential R packages that are widely used in data science
and analysis:

\begin{itemize}
\tightlist
\item
  \textbf{\texttt{dplyr} (part of \texttt{tidyverse}):} For data
  manipulation (filtering, selecting, arranging, summarizing, mutating
  data frames). It provides a consistent and intuitive set of ``verbs''
  for data wrangling.
\item
  \textbf{\texttt{ggplot2} (part of \texttt{tidyverse}):} For data
  visualization. It implements a ``grammar of graphics,'' allowing you
  to build complex and beautiful plots layer by layer.
\item
  \textbf{\texttt{readr} (part of \texttt{tidyverse}):} For fast and
  friendly reading of rectangular data (CSV, TSV, etc.). Often faster
  and more robust than base R \texttt{read.csv()}.
\item
  \textbf{\texttt{tidyr} (part of \texttt{tidyverse}):} For ``tidying''
  data, making it easier to work with. Functions like
  \texttt{pivot\_wider()} and \texttt{pivot\_longer()} are key for
  reshaping data.
\item
  \textbf{\texttt{purrr} (part of \texttt{tidyverse}):} For functional
  programming, making it easier to work with lists and apply functions
  iteratively.
\item
  \textbf{\texttt{lubridate} (part of \texttt{tidyverse}):} For working
  with dates and times.
\item
  \textbf{\texttt{stringr} (part of \texttt{tidyverse}):} For consistent
  and convenient string manipulation.
\item
  \textbf{\texttt{forcats} (part of \texttt{tidyverse}):} For working
  with factors (categorical variables).
\item
  \textbf{\texttt{data.table}:} An extremely fast and memory-efficient
  package for data manipulation, particularly for very large datasets.
  It has a different syntax compared to \texttt{dplyr}.
\item
  \textbf{\texttt{caret}:} For machine learning workflows
  (Classification And REgression Training). Provides a unified interface
  for many machine learning models.
\item
  \textbf{\texttt{Shiny}:} For building interactive web applications
  directly from R.
\end{itemize}

\textbf{The \texttt{tidyverse}:} Many of these commonly used packages
(\texttt{dplyr}, \texttt{ggplot2}, \texttt{readr}, \texttt{tidyr},
\texttt{purrr}, \texttt{stringr}, \texttt{forcats}, \texttt{lubridate})
are part of a larger ecosystem called the \texttt{tidyverse}. Installing
\texttt{install.packages("tidyverse")} will install all of them at once.
Loading \texttt{library(tidyverse)} will load the core
\texttt{tidyverse} packages.

\textbf{Code Snippets (Illustrating \texttt{tidyverse} usage):}

\begin{Shaded}
\begin{Highlighting}[]
\CommentTok{\# Install tidyverse (if you haven\textquotesingle{}t already {-} this can take a few minutes)}
\CommentTok{\# install.packages("tidyverse") \# Uncomment to run}

\CommentTok{\# Load the tidyverse (this loads dplyr, ggplot2, readr, etc.)}
\FunctionTok{library}\NormalTok{(tidyverse)}
\end{Highlighting}
\end{Shaded}

\begin{verbatim}
## -- Attaching core tidyverse packages ------------------------ tidyverse 2.0.0 --
## v dplyr     1.1.4     v purrr     1.0.4
## v forcats   1.0.0     v readr     2.1.5
## v ggplot2   3.5.2     v tibble    3.2.1
## v lubridate 1.9.4     v tidyr     1.3.1
## -- Conflicts ------------------------------------------ tidyverse_conflicts() --
## x dplyr::filter() masks stats::filter()
## x dplyr::lag()    masks stats::lag()
## i Use the conflicted package (<http://conflicted.r-lib.org/>) to force all conflicts to become errors
\end{verbatim}

\begin{Shaded}
\begin{Highlighting}[]
\FunctionTok{print}\NormalTok{(}\StringTok{"tidyverse loaded (dplyr, ggplot2, readr, etc. are now available)."}\NormalTok{)}
\end{Highlighting}
\end{Shaded}

\begin{verbatim}
## [1] "tidyverse loaded (dplyr, ggplot2, readr, etc. are now available)."
\end{verbatim}

\begin{Shaded}
\begin{Highlighting}[]
\CommentTok{\# Example using dplyr: Filtering a built{-}in dataset}
\CommentTok{\# mtcars is a built{-}in dataset in R}
\NormalTok{filtered\_cars }\OtherTok{\textless{}{-}}\NormalTok{ mtcars }\SpecialCharTok{\%\textgreater{}\%}
  \FunctionTok{filter}\NormalTok{(cyl }\SpecialCharTok{==} \DecValTok{4} \SpecialCharTok{\&}\NormalTok{ gear }\SpecialCharTok{==} \DecValTok{4}\NormalTok{) }\SpecialCharTok{\%\textgreater{}\%}
  \FunctionTok{select}\NormalTok{(mpg, hp, wt)}

\FunctionTok{print}\NormalTok{(}\StringTok{"Filtered cars (using dplyr):"}\NormalTok{)}
\end{Highlighting}
\end{Shaded}

\begin{verbatim}
## [1] "Filtered cars (using dplyr):"
\end{verbatim}

\begin{Shaded}
\begin{Highlighting}[]
\FunctionTok{print}\NormalTok{(}\FunctionTok{head}\NormalTok{(filtered\_cars))}
\end{Highlighting}
\end{Shaded}

\begin{verbatim}
##                 mpg hp    wt
## Datsun 710     22.8 93 2.320
## Merc 240D      24.4 62 3.190
## Merc 230       22.8 95 3.150
## Fiat 128       32.4 66 2.200
## Honda Civic    30.4 52 1.615
## Toyota Corolla 33.9 65 1.835
\end{verbatim}

\begin{Shaded}
\begin{Highlighting}[]
\CommentTok{\# Example using ggplot2: Simple plot (won\textquotesingle{}t display in text output)}
\FunctionTok{ggplot}\NormalTok{(mtcars, }\FunctionTok{aes}\NormalTok{(}\AttributeTok{x =}\NormalTok{ mpg, }\AttributeTok{y =}\NormalTok{ hp)) }\SpecialCharTok{+}
  \FunctionTok{geom\_point}\NormalTok{() }\SpecialCharTok{+}
   \FunctionTok{labs}\NormalTok{(}\AttributeTok{title =} \StringTok{"MPG vs HP in mtcars"}\NormalTok{) }\CommentTok{\# Uncomment to create a plot in your R environment}
\end{Highlighting}
\end{Shaded}

\pandocbounded{\includegraphics[keepaspectratio]{lesson_3_Introduction_to_Packages_files/figure-latex/unnamed-chunk-4-1.pdf}}

This lesson provided a comprehensive introduction to R packages,
explaining their importance, how to install and load them, and how to
find their documentation. You also got an overview of some of the most
widely used packages in the R ecosystem, particularly those within the
\texttt{tidyverse}. Packages are the key to unlocking R's full potential
for data analysis and beyond.

\begin{center}\rule{0.5\linewidth}{0.5pt}\end{center}

\textbf{Module 1.4: Functions and Packages} is now complete!

\textbf{Next, we will begin Phase 2: R Programming Fundamentals,
starting with Module 2.1: Data Manipulation with \texttt{dplyr} and
\texttt{tidyr}.}

\end{document}
